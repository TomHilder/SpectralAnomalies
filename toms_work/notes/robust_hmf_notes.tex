
\documentclass[11pt]{article}
\usepackage{amsmath, amssymb, amsthm, bm}
\usepackage[margin=3cm]{geometry}

\title{Robust HMF Notes: proof/sketch of why there \emph{is} an objective}
\author{Tom Hilder}
\date{25/08/25}

\begin{document}
\maketitle

\section*{Setup}

Take measured data $F_{ij}$ with known variances $\sigma_{ij}^2$. 
We model with low rank:
\begin{equation}
    Y_{ij} = a_i^\top g_j,
\end{equation}
as matrices
\begin{align*}
    Y &: N \times M, \\
    A &: N \times K, \\
    G &: K \times M,
\end{align*}
such that
\begin{equation}
    Y \approx AG.
\end{equation}
Assume Gaussian heteroskedastic noise gives $\chi^2$ objective:
\begin{equation}
    \chi^2(A,G) = \sum_{ij} \frac{(Y_{ij} - a_i^\top g_j)^2}{\sigma_{ij}^2}.
\end{equation}
This is bilinear in $(A,G)$ so solvable with alternating least squares.

\subsection*{a-step}
\begin{align*}
    A_i &\leftarrow G_i F_i, \\
    [G_i]_{kk'} &= \sum_{j=1}^m \frac{g_{kj} g_{k'j}}{\sigma_{ij}^2}, \\
    [F_i]_k &= \sum_{j=1}^m \frac{g_{kj} Y_{ij}}{\sigma_{ij}^2}.
\end{align*}
\subsection*{g-step}
\begin{align*}
    G_j &\leftarrow A_j^\top F_j, \\
    [A_j]_{kk'} &= \sum_{i=1}^N \frac{a_{ik} a_{ik'}}{\sigma_{ij}^2}, \\
    [F_j]_k &= \sum_{i=1}^N \frac{a_{ik} Y_{ij}}{\sigma_{ij}^2}.
\end{align*}
This is weighted least squares (WLS) for $a_i$ or $g_j$ given fixed $G$ or $A$.

\section*{Adding Robustness}

Outliers are not dealt with well by $\chi^2$ objective. 
Instead, define
\begin{equation}
    r_{ij} = \frac{Y_{ij} - a_i^\top g_j}{\sigma_{ij}}, \qquad 
    L(A,G) = \sum_{ij} \rho(r_{ij}).
\end{equation}
If $\rho(r) = \tfrac{1}{2} r^2$, we recover the above. 
Switch to a Cauchy likelihood, taking the negative log likelihood as our loss function:
\begin{equation}
    \rho(r) = \frac{c^2}{2} \log\!\left(1 + \left(\frac{r}{c}\right)^2\right).
\end{equation}

\section*{Auxiliary Form}

Claim: the loss can be expressed for any $r>0$ in the following way
\begin{equation}
    \rho(r) = \min_{0<w\leq1} \left[ \tfrac{1}{2} w r^2 + \phi(w) \right],
\end{equation}
where
\begin{equation}
    \phi(w) = \frac{c^2}{2}\,(w - 1 - \log w).
\end{equation}
Let's prove that. Define
\begin{align}
    J\left( w ; r \right) &= \tfrac{1}{2} w r^2 + \tfrac{Q^2}{2} \left( w - 1 - \log{w}\right)
\end{align}
Differentiating with respect to $w$:
\begin{equation}
    \frac{\partial J}{\partial w} 
    = \tfrac{1}{2}r^2 + \frac{c^2}{2}\left(1 - \frac{1}{w}\right),
\end{equation}
and
\begin{align}
    \frac{\partial^2 J}{\partial w^2} = \frac{Q^2}{2} \frac{1}{w^2} > 0, \qquad \forall \, Q,w > 0.
\end{align}
thus we know that the critical point minimises $J$.
Setting $\partial_w J=0$ yields
\begin{align}
    \hat{w}(r) &= {\rm argmin}_w \, J (w ; r) \\
    &= \frac{1}{1 + (r/Q)^2},
\end{align}
and note that $\hat{w} \in (0, 1]$ because $(r/Q)^2 \geq 0$.
We now prove the claim. First set $t = (r/c)^2 \geq$, then
\begin{equation}
    \hat{w} = \frac{1}{1+t}, \qquad r^2 = Q^2 t,
\end{equation}
such that
\begin{align}
    J (\hat{w} ; r) &= \tfrac{1}{2} w r^2 + \phi(\hat{w}) \\
    \Rightarrow \tfrac{1}{2} w r^2 &= \frac{Q^2}{2} \frac{t}{1 + t} \\
    \Rightarrow \phi(\hat{w}) &= \frac{Q^2}{2} \left( \hat{w} - 1 - \log{\hat{w}} \right) \\
    &= \frac{Q^2}{2} \left( -\frac{t}{1 + t} + \log(1 + t) \right) \\
    \Rightarrow J (\hat{w} ; r) &= \frac{Q^2}{2} \left( 1 + \left( \tfrac{r}{Q}\right)^2 \right).
\end{align}
Thus
\begin{align}
    \rho(r) &= J ( \hat{w} ; r ) \\
    &= \min_{0<w\leq1} \left[ \tfrac{1}{2} w r^2 + \phi(w) \right],
\end{align}
as claimed.

\section*{Three-Step Algorithm}

Define new objective:
\begin{equation}
    J(A,G,W) = \frac{1}{2}\sum_{ij} \left[ w_{ij} r_{ij}^2 + \phi(w_{ij}) \right],
\end{equation}
with $r_{ij} = (Y_{ij} - a_i^\top g_j)/\sigma_{ij}$.
By construction,
\begin{equation}
    L(A,G) = \min_W J(A,G,W),
\end{equation}
and if
\begin{align}
    \hat{W} &= {\rm argmin}_w \, J(A, G, W),
\end{align}
then
\begin{align}
    \left[ \hat{W} \right]_{ij} &= \hat{w} (r_{ij}) \\
    &= \frac{1}{1 + (r_{ij} / Q)^2}.
\end{align}
This immediately yield's Hogg's procedure
\begin{align*}
    \text{w-step:} \quad & w_{ij} \leftarrow \hat{w}(r_{ij}), \\
    \text{a-step:} \quad & \text{solve WLS for $A$ with new weights}, \\
    \text{g-step:} \quad & \text{solve WLS for $G$ with new weights}.
\end{align*}
where the a-step optises the quadratic
\begin{align}
    Q(A \, | \, G, W) &= \frac{1}{2} \sum_{ij} w_{ij} r_{ij}^2,
\end{align}
and the g-step optimises $Q(G \, | \, A, W)$.
It should be pretty apparent now that the procedure gives the MLE with a Cauchy likelihood.

\section*{Extra convincing (showing that the procedure optimises $L$)}

Consider one outer cycle starting at $(A^{(t)}, G^{(t)})$.  
Choose $W^{(t)} = \hat{w}(r(A^{(t)},G^{(t)}))$.  
Then
\begin{equation}
    L(A^{(t)}, G^{(t)}) = J(A^{(t)}, G^{(t)}, W^{(t)}).
\end{equation}
With frozen $W^{(t)}$, the a- and g-steps minimize $Q(\cdot \,| \,W^{(t)})$.
Since our total objective is $J = Q + \sum \phi(W^{(t)})$, this implies
\begin{equation}    
    J(A^{(t+1)}, G^{(t+1)}, W^{(t)}) \le J(A^{(t)}, G^{(t)}, W^{(t)}).
\end{equation}
We're guaranteed to be helped by the w-step again now, so setting
\begin{align}
    W^{(t+1)} = \hat{w} \left( r(A^{(t+1)}, G^{(t+1)}) \right),
\end{align}
and using our result from the previous section gives
\begin{equation}
    J(A^{(t+1)}, G^{(t+1)}, W^{(t+1)}) \le J(A^{(t+1)}, G^{(t+1)}, W^{(t)}).
\end{equation}
Thus chaining the inequalities and $L(A,G) = \min_W J(A, G, W)$ gives
\begin{equation}
    L(A^{(t+1)}, G^{(t+1)}) \le L(A^{(t)}, G^{(t)}).
\end{equation}
This is enough to guarantee that robust HMF with Hogg's w-step converges to the Cauchy MLE.

\section*{Extra Notes}

\begin{itemize}
    \item The update $w_{ij} \leftarrow \hat{w}(r_{ij})$ is exactly Hogg's update rule if one combines $\sigma_{ij}$ into the weights: 
    $\tilde{w}_{ij} = w_{ij}/\sigma_{ij}^2$. I kept the updated weights separate to the data variances since it makes the connection to the MLE clearer (in my mind).
    \item Different $\rho$ losses (negative log likelihoods) yield different w-step rules.
    \item This is basically the standard argument for IRLS (iteratively reweighted least squares), didn't require much extra.
    \item I accidentally overloaded my notation a bit ($Q$ is two things). Sorry about that.
    \item This view gives a very natural interpretation for any regularisation, priors or constraints.
\end{itemize}

\end{document}
