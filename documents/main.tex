\documentclass{article}
\usepackage[letterpaper]{geometry}
\usepackage{setspace}
\usepackage{amsmath}

% page layout -- the idea is, set text height and width and then set margins to match
\setlength{\textheight}{9.00in}
\setlength{\textwidth}{5.00in}
\setlength{\topmargin}{0.5\paperheight}\addtolength{\topmargin}{-1in}\addtolength{\topmargin}{-0.5\textheight}\addtolength{\topmargin}{-\headsep}\addtolength{\topmargin}{-\headheight}
\setlength{\oddsidemargin}{0.5\paperwidth}\addtolength{\oddsidemargin}{-1in}\addtolength{\oddsidemargin}{-0.5\textwidth}
\pagestyle{myheadings}
\markboth{foo}{\sffamily Hogg / robust heteroskedastic matrix factorization}

% other text layout adjustment commands
\renewcommand{\newblock}{} % this adjusts the bibliography style.
\setstretch{1.08}
\sloppy\sloppypar\raggedbottom
\frenchspacing

% math macros
\DeclareMathOperator{\solve}{solve}
\DeclareMathOperator{\svd}{svd}

\begin{document}

\section{Introduction}

\paragraph{Prior work:}
\begin{itemize}
    \item At the beginning of time, there was principal components analysis (PCA) \cite{pca}, the last universal common ancestor (LUCA) of all self-supervised (or unsupervised) machine-learning methods.
    PCA replaces (or models) the $N\times M$ rectangular data $Y$ with a rectangular matrix $L$ of rank $K<\min(N,M)$ that minimizes the sum of squares of the residual $Y-L$.
    Like almost all of its machine-learning-method descendants, PCA requires complete, rectangular data; it treats every data point identically.
    It is extremely sensitive to outliers; a single bad pixel in one data record can spoil many or all of the delivered eigenvectors.
    After all, it is a model to minimize unmodeled \emph{variance} (squared error), while the empirical variance in a block of data can easily be dominated by one or a few pixels.
    \item Unrelated to the successes of PCA, humankind evolved, looked at the stars, embraced weighted least squares (chi-squared fitting) \cite{laplace}, got upset about the influences of outliers, and immediately started sigma clipping \cite{sigmaclip}.
    This has been the dominant method for making weighted-least-squares fits insensitive to rare outliers or data anomalies; that is, to make them more \emph{robust}.
    [HOGG: Algorithm??]
    This method is subject to a kind of ``mode collapse'' in which large amounts of data get clipped out and the model just doesn't represent those data at all.
    \item If ancient astronomers had looked at the statistics literature, they would have seen iteratively reweighted least squares (IRLS) \cite{irls}.
    This method is really a family of methods; but there is a form you can write down [HOGG DO THAT HERE] in which IRLS has all the good properties of weighted least squares with sigma-clipping (and a similar kind of anomaly threshold parameter), but is way less prone to mode collapse.
    The IRLS method is a workhorse in many domains; it has been used in astronomy now and then \cite{things}.
    \item A mathematically rigorous but robust empirical model for (representation for?) data is the robust principal components analysis (Robust-PCA) method of Cand\'es et al \cite{candes}.
    This method attempts to describe the rectangular block of data $Y$ as a sum of a low-rank block $L$ and a sparse block $S$.
    Interestingly, the method attempts to make this exact, such that $Y=L+S$ exactly.
    When the Robust-PCA algorithm is iterated for finite time, it comes finitely close [HOGG: CORRECT?].
    Fundamentally, the Robust-PCA is an alternation of a singular value decomposition (with a threshold on the singular values) to make $L$ and an outlier identification (with a threshold on the residual) to make $S$.
    \item Along a separate thread, Tsalmantza \& Hogg introduced heteroskedastic matrix factorization (HMF) \cite{hmf} as a data-driven dimensionality reduction for astronomical spectra (or other kinds of noisy data).
    The HMF model of rectangular data $Y$ is the rectangular matrix $L$ of rank $K<\min(N,M)$ that minimizes the chi-squared residual (weighted sum of squares).
    HMF is a replacement for PCA that does not require complete data (because missing data can be assigned vanishing weights), and it has the satisfying property that every data point is weighted with its associated inverse uncertainty variance, which represents the amount of information it brings.
    However, HMF is still very sensitive to outliers or anomalies in the data, if they are not weighted appropriately.
\end{itemize}
The algorithm that follows---Robust-HMF---is very much a mash-up of HMF \cite{hmf} and Robust-PCA \cite{candes}.
It adds to Robust~PCA data weighting and the ability to handle missing and low-information pixels.
It sacrifices the nice property of Robust-PCA that it explicitly decomposes the data (or the model for the data) into sparse and low-rank components.
It also [HOGG THINKS] can't be expressed precisely as the optimization of a single scalar objective function, which makes it harder to analyze mathematically.
However, empirically, it works very well in standard astronomical contexts, as we will see.

\section{Assumptions and choices}

\section{Method}
The setup is that there are $N$ spectra $y_i$ ($1\leq i\leq N$),
each of which has $M$ pixels $y_{ij}$ ($1\leq j\leq M$).
Importantly for what follows, some of the pixels can have dummy or null values, because not all pixels of all spectra will have been observed in most real contexts.
Along with each pixel value $y_{ij}$ there is an associated inverse-variance weight $w_{ij}$.
Usually these weights will be inverse uncertainty variances $\sigma_{ij}^{-2}$.
Importantly, the pixels with dummy or null values of $y_{ij}$ should have vanishing weights $w_{ij}$.
Vanishing weights ($w_{ij}\rightarrow 0$) are equivalent to infinitely large ``error bars'' ($\sigma_{ij}\rightarrow\infty$).

The weights need to have two properties for the method to be safe:
\textsl{(1)}~All the weights need to be strictly non-negative, and
\textsl{(2)}~each weight $w_{ij}$ needs to have units that are inverse of the square of the units of the pixel value $y_{ij}$.
So if the $y_{ij}$ have units of flux, then the units of the weights $w_{ij}$ need to be inverse flux squared.

The data $y_{ij}$ are rectangular in the following sense:
Each index value $i$ corresponds to one, well-defined spectrum of one star, and
each index value $j$ corresponds to one, well-defined wavelength $\lambda_j$ (which could be spectrograph-frame wavelength or stellar rest-frame wavelength, depending on task).
Every $i,j$ pair has an entry, although many of them will have $w_{ij}=0$ because they are unobserved values.
Because the method will be perfectly insensitive to pixels with $w_{ij}=0$, it doesn't matter what the fill value is for $y_{ij}$, but it makes sense to use something sensible (like zero) for numerical safety.

The method will be unstable unless, for each star $i$, there are many observed wavelengths $j$ ($w_{ij} > 0$ for many pixels $j$, for every $i$).
Similarly, we will need it to be the case that, for each wavelength $j$, there are many observed stars $i$ ($w_{ij} > 0$ for many stars $i$, at every $j$).
Without breaking rectangularity, any stars $i$ or wavelengths $j$ that do not meet these criteria can be dropped from the method prior to start.

The model, conceptually, is that the rectangular data matrix composed of the $y_{ij}$ can be represented, up to noise, as a low-rank matrix plus sparse outliers.
To be slightly more specific,
\begin{align}
    y_{ij} &= \sum_{k=1}^K a_{ik}\,g_{kj} + \text{outliers} + \text{noise} \label{eq:model}~,
\end{align}
where the rank is $K$,
each $a_i$ is a $K$-vector of coefficients $a_{ik}$,
each $g_k$ is an $M$-vector eigenvector of the low-rank model,
and the outliers and noise can't really be distinguished from one another.
That is, the model is low-rank plus outliers plus noise but really it can be seen as just low-rank plus residuals.

HOGG: probabilistic interpretation of chi-squared, and then probabilistic interpretation of IRLS in this context; is there one?

\paragraph{Training:}
Training of this model proceeds by alternating least squares with IRLS mixed in.
Before starting the optimization, the coefficients $a_{ik}$ and components $g_{kj}$ are initialized with a singular-value decomposition (SVD) of the data:
\begin{align}
    [Y]_{ij} &= y_{ij} \\
    U\,S\,V &= \svd(Y) \\
    a_{ik} &\leftarrow [U]_{ik}\,[S]_{kk} \\
    g_{ik} &\leftarrow [V]_{kj} ~,
\end{align}
where $Y$ is the rectangular data matrix,
and the operator $\svd()$ takes the standard singular-value decomposition with $U, V$ unitary and $S$ diagonal.
This initialization is not great, because the data are filled with missing values and the SVD has no understanding of that.

After initialization, the steps in the optimization schedule are the following:
The \textbf{a-step} finds best-fit values for the coefficients $a_{ij}$ given the current guess of the eigenvector components $g_{kj}$:
\begin{align}
    a_i &\leftarrow \solve(A_i, b_i) \label{eq:a-step} \\
    [A_i]_{kk'} &= \sum_{j=1}^M g_{kj}\,w_{ij}\,g_{k'j} \\
    [b_i]_k     &= \sum_{j=1}^M g_{kj}\,w_{ij}\,y_{ij} ~,
\end{align}
where $a_i$ is a $K$-vector of coefficients,
the operator $\solve(A, b)$ returns $A^{-1}\,b$,
$A_i$ is a $K\times K$ matrix with entries $[A_j]_{kk'}$,
and $b_i$ is a $K$-vector with components $[b_i]_k$.

The \textbf{g-step} finds best-fit values for the eigenvector components $g_{kj}$
\begin{align}
    g_j &\leftarrow \solve(A_j, b_j) \label{eq:g-step} \\
    [A_j]_{kk'} &= \sum_{i=1}^N a_{ik}\,w_{ij}\,a_{ik'} \\
    [b_j]_k     &= \sum_{i=1}^N a_{ik}\,w_{ij}\,y_{ij} ~,
\end{align}
where $g_j$ is a $K$-vector of eigenvector components,
$A_j$ is a $K\times K$ matrix with entries $[A_j]_{kk'}$,
and $b_j$ is a $K$-vector with components $[b_j]_k$.

This model has a huge set of degeneracies:
The $a_{ik}$ can be multiplied by a factor, and the $g_{kj}$ can be divided by that same factor.
The orientation of the eigenvectors $g_k$ can be rotated and the coefficients $a_{ik}$ can be de-rotated.
In practice, to break some (but not all) of these degeneracies, after each iteration of the a-step and the g-step,
an SVD is performed on the low-rank matrix
to rotate and rescale the $g_k$ axes back to a standard orientation in the data space.
This looks like the following:
\begin{align}
    [L]_{ij} &= \sum_{k=1}^K a_{ik}\,g_{kj} \\
    U\,S\,V &= \svd(L) \\
    a_{ik} &\leftarrow [U]_{ik}\,[S]_{kk} \\
    g_{ik} &\leftarrow [V]_{kj} ~,
\end{align}
similar to the SVD-based initialization described above, but now acting on the low-rank model rather than the data.
Importantly this operation does not change the model predictions \emph{at all}; it just re-scales, re-orients, and re-orders the eigenvectors.

The \textbf{w-step} updates the weights $w_{ij}$ for robustness:
\begin{align}
    w_{ij} &\leftarrow w^\text{(in)}_{ij}\,\frac{Q^2}{w^\text{(in)}_{ij}\,\Delta_{ij}^2 + Q^2} \label{eq:w-step} \\
    \Delta_{ij} &= y_{ij} - \sum_{k=1}^K a_{ik}\,g_{kj} ~,
\end{align}
where the $w^\text{(in)}_{ij}$ are the original input (investigator-specified) data weights,
$Q$ is the dimensionless soft outlier threshold,
and $\Delta_{ij}$ is the residual of the (current best-fit) model at datum $y_{ij}$
This formula \eqref{eq:w-step} looks magical but it's just a zero-safe version of a standard iteratively reweighted least squares method \cite{irls}.
As the residual $\Delta_{ij}$ gets very large, $w_{ij}$ approaches $Q^2 / \Delta_{ij}^2$;
as the residual gets small, $w_{ij}\approx w^\text{(in)}_{ij}$.
Note that this weight adjustment only makes sense if the weights have units that are the inverse square of the data units.

The a-step, g-step, SVD reorientation, and w-step are iterated to convergence.
Convergence is judged by a dimensionless estimate of the size of the g-step adjustment.
The output of training is the full, converged, $N\times K$ matrix $A$ of coefficients $a_{ik}$ and the full, converged, $K\times M$ matrix $G$ of components $g_{kj}$.

\paragraph{Test time:}
At test time, a new data object $y_\ast$ with $M$ pixel values $y_{\ast j}$ is introduced, with associated weights $w_{\ast j}$, including probably some missing data with vanishing weights.
The a-step and w-step are iterated on this object to convergence, keeping all the components $g_{kj}$ fixed.
Convergence is judged by a dimensionless estimate of the size of the a-step adjustment.
The output of test time is $K$ converged coefficients $a_{\ast k}$, or equivalently the low-rank representation $\sum_k a_{\ast k}\,g_{kj}$.

\paragraph{Implementation notes}
Trimming down under-observed objects and wavelengths?
Working on residuals not data?
Actually output synthetic data not a values?

HOGG: The fastness at training time depends on jax \cite{jax}.
The fastness at test time depends on parallelization.
Currently the parallelization isn't correct; need to batch it maybe to reduce shared-memory overheads?

This method does not [HOGG THINKS] optimize a scalar objective; it is \emph{an algorithm}.
Therefore, it can't just be put into standard ML form and optimized by contemporary optimization methods like CJ, BFGS, or Adam.
[HOGG thinks??]

\raggedright\footnotesize
\bibliographystyle{plain}
\bibliography{rhmf}

\end{document}
